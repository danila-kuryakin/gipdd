\include{settings}

\begin{document}

\include{titlepage}

\tableofcontents
\newpage

\section{Цель работы}

Познакомиться с основами проектирования схемы БД, языком описания сущностей и ограничений БД SQL-DDL.

\section{Программа работы}

\begin{enumerate}
	\item Самостоятельное изучение SQL-DDL.
	\item Создание скрипта БД в соответствии с согласованной схемой. Должны присутствовать первичные и внешние ключи, ограничения на диапазоны значений. Демонстрация скрипта преподавателю. 
	\item Создание скрипта, заполняющего все таблицы БД данными.
	\item Выполнение SQL-запросов, изменяющих схему созданной БД по заданию преподавателя. Демонстрация их работы преподавателю.
\end{enumerate}

\section{Теоретическая информация}

\textbf{Язык SQL} (Structured Query Language) -- язык структурированных запросов. Он позволяет формировать весьма сложные запросы к базам данных. В SQL определены два подмножества языка:

\begin{itemize}
	\item \textbf{SQL-DDL} (Data Definition Language) -- язык определения структур и ограничений целостности баз данных. Сюда относятся команды создания и удаления баз данных; создания, изменения и удаления таблиц; управления пользователями и т.д.
	\item \textbf{SQL-DML} (Data Manipulation Language) -- язык манипулирования данными: добавление, изменение, удаление и извлечение данных, управления транзакциями. Функции SQL-DML определяются первым словом в предложении (часто называемом запросом), которое является глаголом: \code{SELECT} (<<выбрать>>), \code{INSERT} (<<вставить>>), \code{UPDATE} (<<обновить>>), и \code{DELETE} (<<удалить>>). 
\end{itemize}

\section{Выполнение работы}

\subsection{Структура базы данных}

\begin{figure}[H]
	\begin{center}
		\includegraphics[scale=0.6]{../../diagram/diagram.png}
		\caption{Схема модели} 
		\label{pic:pic_name} % название для ссылок внутри кода
	\end{center}
\end{figure}

\subsection{Скрипт создания структуры базы данных}

\lstinputlisting{../sql/table.sql}

\subsection{Скрипт заполнения таблиц тестовыми данными}

\lstinputlisting{../sql/filling.sql}

После внесения дополнительных требований преподавателя была изменена структура базы данных, добавленны таблицы и добавленны тестовые данные.

\subsection{Структура базы данных после изменения}

\begin{figure}[H]
	\begin{center}
		\includegraphics[scale=0.6]{../../diagram/diagram2.png}
		\caption{Схема модели 2} 
		\label{pic:pic_name2} % название для ссылок внутри кода
	\end{center}
\end{figure}

\subsection{Скрипт создания структуры базы данных после изменения}

\lstinputlisting{../sql/table2.sql}

\subsection{Скрипт заполнения таблиц тестовыми данными после изменения}

\lstinputlisting{../sql/filling2.sql}

Во время выполнения 3 лабораторной работы в качастве первичных лю
При выполнение 3 лабораторной работы первичные ключи были типа Integer. В последствие типы первичных лючей были изменены а serial.

\subsection{Скрипт создания структуры базы данны первичный ключ serial}

\lstinputlisting{../sql/table3.sql}

\subsection{Скрипт заполнения таблиц тестовыми данными первичный ключ seria}

\lstinputlisting{../sql/filling3.sql}

\section{Выводы}

В ходе выполнения данной работы были изучены основы создания скриптов на языке SQL. С помощью SQL-DDL описаны структуры разрабатываемой схемы базы данных. С использованием SQL-DML созданные структуры заполнены тестовыми данными. Изучен синтаксис обновления структуры существующей таблицы.

\end{document}
