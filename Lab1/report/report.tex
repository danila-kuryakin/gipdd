\include{settings}

\begin{document}	% начало документа

\include{titlepage}


% Содержание
\tableofcontents
\newpage



\section{Цель работы}
Знакомство со средствами проектирования модели БД, основными типами данных, используемых в проектировании БД.


\section{Программа работы}
\begin{enumerate}
	\item Создание проекта для работы в GitLab.
	\item Выбор задания (предметной области), описание набора данных и требований к хранимым данным в свободном формате в Wiki своего проекта в GitLab.
	\item Формирование в свободном формате (предпочтительно в виде графической схемы) схемы БД, соответствующей заданию. Должно получиться не менее 7 таблиц.
	\item Согласование с преподавателем схемы БД. Обоснование принятых решений и соответствия требованиям выбранного задания. 
	\item Выкладывание схемы БД в свой проект в GitLab.
	\item Демонстрация результатов преподавателю.
\end{enumerate}


\section{Ход выполнения работы}

\subsection{Выбор предметной области}

В качестве задания была выбрана тема "ГИБДД" и определены правила:

\begin{itemize}
\item Есть человек которы может являться как простым водителем так и сотрудником ГИБДД.
\item У человека может быть транспортное средство ТС.
\item Ездя он может нарушать правила дорожного движения (превышение скорости, неправильная парковка, пересечение сплошной полосы, проезд на красный).
\item За нарушения начисляет ся штраф в размере указанной в таблице суммы.


\end{itemize}


\subsection{Структура модели}

Были определены следующие таблицы:
\begin{enumerate}

\item Таблица people хранит имя, фамилию и отчество человека. Содержит следующие атрибуты id, first_name, last_name, middle_name. Пример id - 1,first_name - Данила, last_name - Курякин, middle_name - Александрович. 

\item Таблица driver_license водительские права. Содержит следующие атрибуты id, number номер водительского удостоверения, categories категории которые доступны водителю, categories категории водителя, data_and_time_of_issue, end_date_and_time, unit_gipdd и people_id ссылается на id человека в таблици people. 

\item Таблица inspector. Содержит следующие атрибуты id, police_certificate номер удостоверения, rank звание инспектора, first_name, last_name, middle_name и people_id ссылается на id человека в таблици people.

\item Таблица car. Содержит следующие атрибуты: id, registration_plate номер машины, brand_and_model ссылается на id в таблици machine_directory, categories ссылается на id в таблице dir_categories.

\item Таблица fine. Содержит следующие атрибуты: id, registration_plate номер ТС, driver_license водительское удостоверение, ссылается на атрибут id таблицы driver_license, police_certificate удостоверение полицейского, ссылается на атрибут id таблицы inspector, data_and_time дата и время нарушения, id_violation номер нарушения в справочнике, ссылается на справочник штрафов violation.

\item Таблица machine_directory справочник автомобилей. Содержит следующие атрибуты: id, brand, model.

\item Таблица violation справочник штрафов. Содержит следующие атрибуты: id, title название штрафа, punishment наказание за нарушение.

\item Таблица dir_categories справочник категорий. Содержит следующие атрибуты: id, name название категории.

\item Таблица categories нужна ля связи категорий водителя со справочником категорий. Содержит следующие атрибуты: categories ссылается на categories таблицы driver_license, id_categories на id таблицы dir_categories.


\end{enumerate}


\begin{figure}[H]
	\begin{center}
		\includegraphics[scale=0.6]{../../diagram/diagram.png}
		\caption{Схема модели} 
		\label{pic:pic_name} % название для ссылок внутри кода
	\end{center}
\end{figure}

\section{Выводы}

В ходе выполнения работы была разработана ER-диаграмма модели данных, структура которой согласовывалась с преподавателем.

\end{document}
